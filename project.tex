\documentclass[11pt]{amsart}

\usepackage{amsmath,amssymb,graphicx,bbm}
\usepackage{amsthm,verbatim}
\usepackage{mathrsfs,mathtools}

\usepackage[footnotesize,bf]{caption}
\usepackage[left=1.1in,right=1.1in,top=1in]{geometry}

\usepackage{mathdefs}

%% Patch for amsart date
\usepackage{etoolbox}
\makeatletter
\patchcmd{\@maketitle}
  {\ifx\@empty\@dedicatory}
  {\ifx\@empty\@date \else {\vskip3ex \centering\footnotesize\@date\par\vskip1ex}\fi
   \ifx\@empty\@dedicatory}
  {}{}
\patchcmd{\@adminfootnotes}
  {\ifx\@empty\@date\else \@footnotetext{\@setdate}\fi}
  {}{}{}
\makeatother
%%

\title{Project 1: Finite difference methods}
\author{Akil Narayan}
\date{\today}

\begin{document}
\maketitle

\textbf{1.} (Finite difference methods in 1D)\\
Consider the ordinary differential equation:
\begin{align}\label{eq:ode}
  \ddx{x} \left(\kappa(x) \ddx{x} u(x) \right) &= f(x), & x &\in [0,1],
\end{align}
with homogeneous Dirichlet boundary conditions, $u(0) = u(1) = 0$ , and where scalar diffusion coefficient $\kappa$ is given by,
  \begin{align*}
    \kappa(x) &= 2 + \sum_{\ell=1}^{5} \frac{1}{\ell+1} \sin( \ell \pi x ).
  \end{align*}
  The goal of this exercise will be to numerically compute solutions to this problem. \\[8pt]
  \begin{itemize}
    \item[(a)] Define the operator,
      \begin{align*}
        \widetilde{D}_0 u(x_j) &= \frac{u(x_j + h/2) - u(x_j - h/2)}{h/2}, & h &= 1/(N+1), & x_j &\coloneqq j h,
      \end{align*}
      for a fixed number of points $N \in \N$. Then with $u_j$ the numerical solution approximating $u(x_j)$ for solving the $d=1$ version of \eqref{eq:ode}, consider the scheme,
      \begin{align}\label{eq:D0-def}
        \widetilde{D}_0 \left( \kappa(x_j) \widetilde{D}_0 u_j \right) &= f(x_j), & j \in [N].
      \end{align}
      Show that, for smooth $u$ and $\kappa$, this scheme has second-order local truncation error.
    \item[(b)] Construct an exact solution via the \textit{mathed of manufactured solutions}: posit an exact (smooth) solution $u(x)$ (that satisfies the boundary conditions!) and, compute $f$ in \eqref{eq:ode} so that your posited solutions satifies \eqref{eq:ode}. 
    \item[(c)] Implement the scheme above for solving \eqref{eq:ode}, setting $f$ to be the function identified in part (b), so that you know the exact solution. Show that indeed you achieve second-order convergence in $h$ (say in the $h^{d/2}$-scaled vector $\ell^2$ norm) . (To ``show'' this, plot on a log scale the error as a function of a discretization parameter, such as $h$ or $N$, and verify that the slope of the resulting line is what is expected.)
  \end{itemize}

\textbf{2.} (Finite difference methods in 2/3D)\\
  Consider the following partial differential equation that generalizes \eqref{eq:ode}:
  \begin{align}\label{eq:laplace}
    \nabla \cdot \left(\kappa(\bs{x}) \nabla \bs{u}(\bs{x}) \right) &= f(\bs{x}), & \bs{x} \in [0,1]^d,
  \end{align}
  again with homoegenous Dirichlet boundary conditions, $u\big|_{\partial [0,1]^d} = 0$. Set the diffusion coefficient to be,
  \begin{align*}
    \kappa(\bs{x}) &= 2 + \sum_{k,\ell = 1}^3 \frac{1}{(k+1)(\ell+1)} \sin(\ell \pi x_1) \sin (k \pi x_2), & \bs{x} &= (x_1, x_2)^T.
  \end{align*}
  This problem involves numerically solving the PDE above.
  \begin{itemize}
    \item[(a)] Consider $d = 2$. To discretize the $\nabla$ operator for $d=2$, $\bs{x} = (x_1, x_2)^T$, use,
    \begin{align*}
      \nabla \sim \left(\begin{array}{c} \widetilde{D}_{0,1} \\ \widetilde{D}_{0,2} \end{array}\right),
    \end{align*}
      where $\widetilde{D}_{0,1}$ and $\widetilde{D}_{0,1}$ are one-dimensional versions of \eqref{eq:D0-def} operating in the $x_1$ and $x_2$ directions, respectively. Use the method of manufactured solutions to define an appropriate $f$ so that you know the exact solution. Verify expected order of accuracy (say in $h$) as in the previous problem. What novel practical aspects arise in the two-dimensional case compared to the 1D case?
    \item[(b)] Can you extend your solver to three dimensions? Do you still observe high-order convergence?  Note that in either 2 or 3 dimensions, you may want to consider iterative methods for solving the linear system. (Does the matrix $\bs{A}$ in your linear system have special properties or structure?) Note also that for these problems, if $\bs{u}$ is a vector containing the degrees of freedom for the solution $u$, then you can evaluate $\bs{u} \mapsto \bs{A} \bs{u}$ \textit{without} forming the full $d$-dimensional $\bs{A}$ matrix, and instead using only ``one-dimensional'' versions of $\bs{A}$.
  \end{itemize}

  \textbf{3.} (Finite difference methods for time-dependent problems)\\
  Consider the PDE,
  \begin{align*}
    u_t + a u_x &= 0, & u(x,0) &= \exp(\sin 2 \pi x), & x \in [0, 1),
  \end{align*}
  with periodic boundary conditions, where $k$ is the timestep. In this problem, we'll use the following \textit{Lax-Wendroff} scheme to numerically solve this PDE:
  \begin{align*}
    D^0 u_j^n = -a D_0 u^j_n + \frac{a^2 k}{2} D_+ D_- u_j^n.
  \end{align*}
  \begin{itemize}
    \item[(a)] Show that this scheme has local truncation error that is order $h^2$ in space and $k^2$ in time.
    \item[(b)] Compute the stability bound relating $k$ and $h$ via von Neumann stability analysis.
    \item[(c)] Implement the Lax-Wendroff scheme (say with $a = 1$ and integrating up to time $T=1$) and numerically verify that the scheme is second-order in space, and second-order in time.
  \end{itemize}

%\bibliographystyle{siam}
%\bibliography{references}

\end{document}
